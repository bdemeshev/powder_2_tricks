% arara: xelatex
\documentclass[12pt]{article}

% \usepackage{physics}

\usepackage{hyperref}
\hypersetup{
    colorlinks=true,
    linkcolor=blue,
    filecolor=magenta,      
    urlcolor=cyan,
    pdftitle={Overleaf Example},
    pdfpagemode=FullScreen,
    }

\usepackage{verse}

\usepackage{tikzducks}

\usepackage{tikz} % graphics in tikz
\usetikzlibrary{shapes, arrows, positioning}
\usepackage{microtype} % punctuation hanging

\usepackage{array} % for fixed-width columns

\usepackage{indentfirst} % indent in first paragraph

\usepackage{sectsty} % for centering section titles
\allsectionsfont{\centering}

\usepackage{amsmath, amsfonts, amssymb} % standard math packages

\usepackage{comment}

\usepackage[top=2cm, left=1.2cm, right=1.2cm, bottom=2cm]{geometry} % page text size

\usepackage{lastpage} % to find the last page number

\usepackage{enumitem} % additional list features
%  for example \begin{enumerate}[resume] allows continuing numbering in a new list
\usepackage{caption}

\usepackage{url} % to use \url{link to web}



\newcommand{\smallduck}{\begin{tikzpicture}[scale=0.3]
    \duck[
        cape=black,
        hat=black,
        mask=black
    ]
    \end{tikzpicture}}

\usepackage{fancyhdr} % fancy headers and footers
\pagestyle{fancy}
\lhead{}
\chead{Local extremum without derivatives}
\rhead{}
\lfoot{}
\cfoot{}
\rfoot{}

\renewcommand{\headrulewidth}{0.4pt}
\renewcommand{\footrulewidth}{0.4pt}

\usepackage{tcolorbox} % boxes!

\usepackage{todonotes} % for inserting notes about what remains to be done
% \todo{Here coefficients need to be fixed}
% \missingfigure{Here will be The Last Day of Pompeii}
% \listoftodos - prints all \todo notes


% prettier tables
\usepackage{booktabs}
% commandments from documentation:
% 1. Don't use vertical lines
% 2. Don't use double lines
% 3. Units of measurement - in table header
% 4. Don't abbreviate .1 instead of 0.1
% 5. Repeat recurring values, don't say "same"


\setcounter{MaxMatrixCols}{20}
% by crazy default pmatrix supports only 10 cols :)


\usepackage{fontspec}
\usepackage{libertine}
\usepackage{polyglossia}


\usepackage[bibencoding=auto, backend=biber, sorting=none, style=alphabetic]{biblatex}
\addbibresource{poly_opt_tricks.bib}


\setmainlanguage{english}
\setotherlanguages{russian}

% download "Linux Libertine" fonts:
% http://www.linuxlibertine.org/index.php?id=91&L=1
% \setmainfont{Linux Libertine O} % or Helvetica, Arial, Cambria
% why do we need \newfontfamily:
% http://tex.stackexchange.com/questions/91507/
% \newfontfamily{\cyrillicfonttt}{Linux Libertine O}

%% econometric abbreviations
\DeclareMathOperator{\Cov}{\mathbb{C}ov}
\DeclareMathOperator{\Corr}{\mathbb{C}orr}
\DeclareMathOperator{\Var}{\mathbb{V}ar}
\DeclareMathOperator{\col}{col}
\DeclareMathOperator{\row}{row}

\let\P\relax
\DeclareMathOperator{\P}{\mathbb{P}}

\DeclareMathOperator{\E}{\mathbb{E}}
% \DeclareMathOperator{\tr}{trace}
\DeclareMathOperator{\card}{card}
\DeclareMathOperator{\mgf}{mgf}

\DeclareMathOperator{\Convex}{Convex}
\DeclareMathOperator{\plim}{plim}

\DeclareMathOperator{\yan}{yan}

\usepackage{mathtools}
\DeclarePairedDelimiter{\norm}{\lVert}{\rVert}
\DeclarePairedDelimiter{\abs}{\lvert}{\rvert}
\DeclarePairedDelimiter{\scalp}{\langle}{\rangle}
\DeclarePairedDelimiter{\ceil}{\lceil}{\rceil}

\newcommand{\cN}{\mathcal{N}}
\newcommand{\cF}{\mathcal{F}}

\newcommand{\RR}{\mathbb{R}}
\newcommand{\NN}{\mathbb{N}}
\newcommand{\hb}{\hat{\beta}}
\newcommand{\dPois}{\mathrm{Pois}}





\begin{document}


\begin{verse}
    \begin{flushright}
        — What do we know about the fox? \\
        — Nothing. And not even all of that. \\

        Boris Zakhoder
    \end{flushright}
\end{verse}

The goal of this note is to explain how to find local extrema of polynomials without derivatives.
The first part presents the method of Gleb Veselsky, invented by him at summer camp in August 2024.
Gleb's method is based on selecting additional factors for the arithmetic-geometric mean inequality.
The method works for polynomials of any degree, factored into linear factors.
The second part presents the method of Johannes Hudde, mayor of Amsterdam in the 17th century. 
Hudde's method is based on identifying multiple roots by multiplying polynomial coefficients by an arithmetic progression.


\section{Method of means for finding extrema}

\begin{tcolorbox}[colback=yellow!50!red!25!white]
The arithmetic mean of positive numbers is greater than or equal to the geometric mean,
\[
\frac{x_1 + x_2 + \dots + x_n}{n} \geq (x_1 x_2 \dots x_n)^{1/n}.
\]
Equality is achieved if and only if
\[
x_1 = x_2 = \dots = x_n.
\]
\end{tcolorbox}

\subsection*{Simple example}

Find the local maximum of the function $f(x) = x^2 \cdot (6 - 2x)$.


Solution. Note that $x + x + (6 - 2x) = 6$.
Therefore, we know the arithmetic mean,
\[
\frac{x + x + (6- 2x)}{3} = 2.
\]
By the inequality of means,
\[
\frac{x + x + (6- 2x)}{3} \geq (x\cdot x \cdot (6 - 2x))^{1/3}.
\]
Hence, if $x > 0$ and $6 - 2x>0$, 
\[
2^3 \geq x^2 \cdot (6 - 2x).
\]
The maximum value is achieved when $x = 6 - 2x$, that is when $x=2$.


TODO: add picture


\subsection*{More complex example}


Find the local maximum of the function $f(x) = x (x + 1) (2 - x)$.


Houston, we have problems with the old solution!
In the previous solution we were lucky twice.
First, in the simple problem the sum of factors happened to be constant.
Second, in the simple problem simultaneous equality of all factors was possible. 
In the new problem neither is the sum equal to a constant, $x + (x + 1) + (2 - x) = 3 +x$, nor is simultaneous equality $x = x+1 = 2-x$ possible.

However, the optimum point will not change if we multiply one factor by $a > 0$ and another by $b > 0$.
By choosing $a$ and $b$ we can restore the double luck!

Let's find the maximum of the function $h(x) = ax \cdot b(x + 1) (2 - x)$.

We require a constant sum and equality of factors!
\[
\begin{cases}
    a + b - 1 = 0 \\
    ax = 2 - x \\
    b(x + 1) = 2 - x.
\end{cases}
\]

The quantities $a$ and $b$ play an auxiliary role, let's eliminate them, $a = (2 - x)/x$, $b = (2 - x)/(x + 1)$!
We get one equation 
\[
\frac{2 - x}{x} + \frac{2 - x}{x + 1} - 1 = 0.
\]

Multiply by the denominators,
\[
(2 - x)(x + 1) + (2 - x) x - x (x + 1) = 0.
\]

The reader familiar with derivatives may notice that we obtained the derivative!
\[
f'(x) = (2 - x)(x + 1) + (2 - x) x - x (x + 1).
\]

It remains only to solve the quadratic equation,
\[
(2 - x)(x + 1) + (2 - x) x - x (x + 1) = 0.
\]

We get $x_1 = (1 - \sqrt{7}) / 3$, $x_2 = (1 + \sqrt{7}) / 3$ — points suspicious of being extrema.


Similarly, one can find local extrema of any polynomial factored into factors. 
To find the extremum of an arbitrary polynomial of degree $n$, $n - 1$ additional factors are needed. 
And special cases with luck can be solved faster. 

TODO: add picture


\subsection*{Lucky example}

Find the local maximum of the function $f(x) = x^{10} (5 - x)$.

Multiply the function by $10$ to make the sum of factors constant.
\[
h(x) = x^{10}(50 - 10x).
\]

Using the inequality of means, valid when $x > 0$ and $50 - 10x > 0$,
\[
\frac{x + x + \dots + x + (50 - 10x)}{11} \geq (x \cdot x \cdot \dots \cdot x \cdot (50 - 10x) )^{1/11}.
\]
Hence, when $x > 0$ and $50 - 10x > 0$,
\[
h(x) \leq (50/11)^{11}.
\]
The maximum is achieved when $x = x = \dots = x = 50 - 10x$, that is $x = 50/11$.

TODO: add picture

\subsection*{Another example}

Let's find the extrema of the function 
\[
f(x) = 4x + 3 + \frac{1}{x - 1}.
\]
For convenience, replace $x - 1$ with $t$,
\[
g(t) = 4(t + 1) + 3 + \frac{1}{t} = 7 + 4t + \frac{1}{t}.
\]
By the inequality of means, we find the local minimum,
\[
\frac{4t + 1/t}{2} \geq \sqrt{4t \cdot 1/t} = 2.
\]
Equality is achieved when $4t = 1/t$, that is when $t = 1/2$ and $t = -1/2$.
The point $t=1/2$ or $x = 1.5$ is a local minimum, and $t = -1/2$ or $x = 0.5$ is a local maximum.


TODO: add picture


\section{Johannes Hudde's method with arithmetic progressions}

Johannes Hudde lived in the 17th century, was a mathematician, mayor of Amsterdam, and invented a method for finding local extrema without derivatives. 

Again, let's start with a simple problem!

Find at least one local extremum of the function
\[
f(x)  = (x - 2)^2 (x - 5) (x - 7)^4.
\]

At the point $x=5$ there is no extremum, since when the variable $x$ passes through $5$, the function changes sign. 
But at points $x = 2$ and $x = 7$ there are local minima. 
At these points the function equals zero, and to the right and left of these points the function is positive. 
The function achieves its local minimum $f_{\text{min}} = 0$ at points of multiple factors!


TODO: add picture


However, the extremum is not necessarily equal to zero. 
It can equal some number $m$.
Therefore, Johannes Hudde declared a hunt for multiple roots of the equation $f(x) = m$.


\subsection*{Hunt for multiple roots}

Let's devise an operation that would simplify the polynomial $p(x) = f(x) - m$, 
but at the same time any multiple root of the polynomial $p(x)$ would remain a root of our new invented polynomial.

Let's start with the parabola $p(x) = (x - 1)^2 = x^2 - 2x + 1$. 
Of course, the multiple root of this parabola is $x = 1$.


I don't know how Johannes Hudde came to his idea. 
Probably it was long hours of fruitless searching, and then sudden discovery. 
What follows is only my attempt to reconstruct his reasoning. 

We need to build a new polynomial $q(x)$ based on the old $p(x) = x^2 - 2x + 1$.
So let's multiply the coefficients of the old one by arbitrary constants, $q(x) = ax^2 - 2bx + c$. 
Hmm, but with this operation we can, of course, lose the multiple root! 
So let's bring the multiple root back! 
For this, the number $c$ cannot be arbitrary, it must be chosen so that the multiple root doesn't die. 
In our toy example we know that the multiple root equals $x = 1$.
Therefore the new polynomial must have the form $q(x) = a(x - 1)^2 - 2b (x - 1)$.

Expand the brackets!
\[
    q(x) = a(x - 1)^2 - 2b (x - 1) = ax^2 - 2(a + b) x + (a + 2b).
\]
Look carefully at the coefficients: $a$, $a + b$, $a + 2b$!
They form an arithmetic progression!

Perhaps this is exactly how the theorem was discovered.

Theorem (Johannes Hudde):
If the coefficients of the original polynomial are multiplied by any arithmetic progression,
then the roots of the new polynomial will necessarily include all multiple roots of the original.

We have, of course, proved it only for a quadratic polynomial with root $x=1$, but let's believe that it's true for any polynomial, and apply it in practice!

Find the local maximum of the function $f(x) = x (x + 1) (2 - x)$.

Expand the brackets, $f(x) = -x^3 + x^2 + 2x$.

We don't know what the extremum equals, so we'll look for multiple roots of the polynomial 
\[
p(x) = -x^3 + x^2 + 2x - m.
\]
Multiply the coefficients of this polynomial by the progression $(3, 2, 1, 0)$, to get rid of the unknown $m$,
\[
\yan(p(x)) = -3x^3 + 2x^2 + 2x + 0.
\]
Solve the equation, $-3x^3 + 2x^2 + 2x + 0 = 0$, we get roots $x_1 = 0$,  $x_2 = (1 - \sqrt{7}) / 3$, $x_3 = (1 + \sqrt{7}) / 3$.

Root $x = 0$ is possible only when $m = 0$, with $p(x) = x\cdot (-x^2 + x + 2)$ and root $x = 0$ is not multiple. 

The reader familiar with derivatives may notice that multiplication by the arithmetic progression $\dots, 3, 2, 1, 0$ and canceling the factor $x$ is completely equivalent to taking the derivative. 
\[
\yan(p(x)) = -3x^3 + 2x^2 + 2x + 0 = x \cdot (-3x^2 + 2x + 2) = x \cdot p'(x).
\]

The power of Johannes Hudde's method is that we can choose the arithmetic progression ourselves. 
And we can specifically put zero in the progression where there is an unpleasant coefficient in the original polynomial.
More details about the method can be read in the wonderful article by Jeff Uzuki \cite{uzuki2005lost}.

\subsection*{Finding a tangent}

Find the tangent to the graph $x^3$ at the point $x = 2$.

We want root $x = 2$ to be a multiple root of the equation $x^3 = kx + b$.

Move everything to the left side 
\[
p(x)  = x^3 + 0x^2 - kx - b = 0.
\]
Multiply by two different progressions, $(2, 1, 0, -1)$ and $(3, 2, 1, 0)$.
The first progression will rid us of parameter $k$, and the second of $b$.
By the way, multiplication by progression $(0, 1, 2, 3)$ would rid us of high powers.
\[
\begin{cases}
    2x^3 + b = 0 \\
    3x^3 - kx = 0 \\
\end{cases}
\]
We want the multiple root to be $x = 2$.
From the system we find $b = - 16$, $k = 12$.


\subsection*{Another example}

Let's find the extrema of the function 
\[
f(x) = 4x + 3 + \frac{1}{x - 1}.
\]
For convenience, replace $x - 1$ with $t$,
\[
g(t) = 4(t + 1) + 3 + \frac{1}{t} = 4t + 7 + \frac{1}{t}.
\]
Set this function equal to the still unknown extremum value $m$,
\[
4t + 7 + \frac{1}{t} = m.
\]
Multiply by $t$, 
\[
4t^2 + (7 - m) t + 1 = 0.
\]
This equation should have a multiple root, multiply coefficients by progression $(1, 0, -1)$,
\[
4t^2 - 1 = 0.
\]
Hence, $t = 1/2$ and $t = -1/2$.
Therefore, $x = 1.5$ and $x = 0.5$.


\subsection*{Proof of Johannes Hudde's theorem}

First, let's see how the operation $\yan()$ of multiplying polynomial coefficients by an arithmetic progression 
acts on the simplest polynomials with multiple root $x_0$:
\[
\yan(x^n(x - x_0)^2) = \yan(x^{n+2} - 2x^{n+1} x_0 + x^n x_0^2) = ax^{n+2} - 2(a + b)x^{n+1} x_0 + (a+2b)x^n x_0^2
\]
We substitute the multiple root $x = x_0$ of the original polynomial into the transformed polynomial and notice that we get zero. 

Second, multiplying polynomial coefficients by an arithmetic progression, $\yan()$, is a linear operation:
\[
\yan(a(x) + b(x)) = \yan(a(x)) + \yan(b(x)).
\]
Third, any polynomial with multiple root $x_0$ can be written as
\[
p(x) = a_n x^n (x - x_0)^2 + a_{n-1} x^{n-1}(x - x_0)^2 + \dots + a_0 (x - x_0)^2.
\]
Apply operation $\yan()$ to the left and right sides. 
By linearity we apply $\yan()$ to each term.
One factor $(x - x_0)$ survives in each term, and therefore in the entire sum the root that was multiple in the original polynomial survives.


\section*{Sources of wisdom}
\addcontentsline{toc}{section}{Sources of wisdom}
\printbibliography[heading=none]





\end{document}