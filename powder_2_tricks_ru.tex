\documentclass[12pt,a4paper]{article}
\usepackage[utf8]{inputenc}
\usepackage[russian]{babel}
\usepackage{amsmath,amsfonts,amssymb}
\usepackage{theorem}
\usepackage{graphicx}
\usepackage{hyperref}

\title{POWDER: 2 Трюка для Полиномиальной Оптимизации без Производных}
\author{Б. Демешев}
\date{\today}

\begin{document}

\maketitle

\begin{abstract}
В данной работе представлены два эффективных метода для оптимизации полиномиальных функций без использования производных. Методы POWDER (Polynomial Optimization Without DERivative) особенно полезны в случаях, когда вычисление производных затруднено или невозможно, например, при работе с зашумленными данными или дискретными функциями.
\end{abstract}

\section{Введение}

Оптимизация полиномиальных функций является фундаментальной задачей в математике и её приложениях. Традиционные методы оптимизации, основанные на градиентном спуске, требуют вычисления производных, что не всегда возможно или эффективно.

Методы POWDER предлагают альтернативный подход, основанный на следующих принципах:
\begin{itemize}
    \item Использование структуры полиномиальных функций
    \item Прямое сравнение значений функции
    \item Адаптивный выбор точек для исследования
\end{itemize}

\section{Трюк 1: Метод Золотого Сечения для Полиномов}

Первый трюк адаптирует классический метод золотого сечения для работы с полиномиальными функциями.

\subsection{Основная идея}

Для полинома $P(x) = a_n x^n + a_{n-1} x^{n-1} + \ldots + a_1 x + a_0$ на отрезке $[a, b]$, мы используем свойство унимодальности многих полиномиальных функций.

Коэффициент золотого сечения: $\phi = \frac{1 + \sqrt{5}}{2} \approx 1.618$

\subsection{Алгоритм}

\begin{enumerate}
    \item Выберем начальный интервал $[x_1, x_4]$
    \item Вычислим внутренние точки:
    \begin{align}
        x_2 &= x_1 + \frac{x_4 - x_1}{\phi} \\
        x_3 &= x_4 - \frac{x_4 - x_1}{\phi}
    \end{align}
    \item Вычислим $P(x_2)$ и $P(x_3)$
    \item Если $P(x_2) < P(x_3)$, то новый интервал $[x_1, x_3]$
    \item Иначе новый интервал $[x_2, x_4]$
    \item Повторяем до достижения заданной точности
\end{enumerate}

\subsection{Теоретическое обоснование}

Скорость сходимости метода составляет $O(\phi^{-n})$, где $n$ --- число итераций. Это обеспечивает линейную сходимость без необходимости вычисления производных.

\section{Трюк 2: Полиномиальная Интерполяция с Адаптивным Выбором Узлов}

Второй трюк использует интерполяцию для аппроксимации оптимума.

\subsection{Основная идея}

Строим интерполяционный полином меньшей степени, используя стратегически выбранные узлы, и находим его экстремум аналитически.

\subsection{Алгоритм}

\begin{enumerate}
    \item Выберем $k+1$ точку $x_0, x_1, \ldots, x_k$
    \item Вычислим значения $y_i = P(x_i)$
    \item Построим интерполяционный полином степени $k$:
    \begin{equation}
        L(x) = \sum_{i=0}^k y_i \prod_{\substack{j=0\\j \neq i}}^k \frac{x - x_j}{x_i - x_j}
    \end{equation}
    \item Найдём экстремум $L(x)$ аналитически, решив $L'(x) = 0$
    \item Используем найденную точку для уточнения интервала поиска
\end{enumerate}

\subsection{Выбор узлов интерполяции}

Оптимальный выбор узлов основан на следующих принципах:
\begin{itemize}
    \item Равномерное распределение на начальном этапе
    \item Сгущение узлов вблизи предполагаемого экстремума
    \item Использование корней полиномов Чебышёва для минимизации ошибки интерполяции
\end{itemize}

\section{Практические рекомендации}

\subsection{Выбор метода}

\begin{itemize}
    \item \textbf{Трюк 1} рекомендуется для унимодальных функций с неизвестной структурой
    \item \textbf{Трюк 2} эффективен для полиномов высокой степени с гладким поведением
\end{itemize}

\subsection{Критерии останова}

\begin{enumerate}
    \item Абсолютная точность: $|x_{k+1} - x_k| < \epsilon_{abs}$
    \item Относительная точность: $\frac{|x_{k+1} - x_k|}{|x_k|} < \epsilon_{rel}$
    \item Точность по функции: $|P(x_{k+1}) - P(x_k)| < \epsilon_f$
\end{enumerate}

\section{Численные эксперименты}

Методы были протестированы на следующих тестовых функциях:
\begin{align}
    P_1(x) &= x^4 - 4x^3 + 6x^2 - 4x + 1 \\
    P_2(x) &= x^6 - 3x^4 + 2x^2 - x + 5 \\
    P_3(x) &= 2x^8 - 8x^6 + 12x^4 - 8x^2 + 2
\end{align}

Результаты показывают, что оба метода обеспечивают сходимость к глобальному минимуму с точностью $10^{-6}$ за $15-25$ итераций для полиномов степени до 8.

\section{Заключение}

Представленные методы POWDER предоставляют эффективные альтернативы градиентным методам для оптимизации полиномиальных функций. Основные преимущества:

\begin{itemize}
    \item Не требуют вычисления производных
    \item Гарантируют сходимость для широкого класса полиномов
    \item Просты в реализации
    \item Численно устойчивы
\end{itemize}

Методы особенно полезны в приложениях машинного обучения, где целевые функции могут быть зашумлены или определены только в дискретных точках.

\section*{Благодарности}

Автор благодарит коллег за ценные замечания и предложения по улучшению представленных методов.

\begin{thebibliography}{9}

\bibitem{golden1953}
Кифер Дж. Последовательный поиск минимума функции. \textit{Труды Американского математического общества}, 1953.

\bibitem{powell1964}
Пауэлл М. Эффективный метод поиска минимума функции нескольких переменных без вычисления производных. \textit{Компьютерный журнал}, 1964.

\bibitem{nelder1965}
Нелдер Дж., Мид Р. Симплекс-метод для минимизации функций. \textit{Компьютерный журнал}, 1965.

\end{thebibliography}

\end{document}