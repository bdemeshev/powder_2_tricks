% arara: xelatex
\documentclass[12pt]{article}

% \usepackage{physics}

\usepackage{hyperref}
\hypersetup{
    colorlinks=true,
    linkcolor=blue,
    filecolor=magenta,      
    urlcolor=cyan,
    pdftitle={Overleaf Example},
    pdfpagemode=FullScreen,
    }

\usepackage{verse}

\usepackage{tikzducks}

\usepackage{tikz} % картинки в tikz
\usetikzlibrary{shapes, arrows, positioning}
\usepackage{microtype} % свешивание пунктуации

\usepackage{array} % для столбцов фиксированной ширины

\usepackage{indentfirst} % отступ в первом параграфе

\usepackage{sectsty} % для центрирования названий частей
\allsectionsfont{\centering}

\usepackage{amsmath, amsfonts, amssymb} % куча стандартных математических плюшек

\usepackage{comment}

\usepackage[top=2cm, left=1.2cm, right=1.2cm, bottom=2cm]{geometry} % размер текста на странице

\usepackage{lastpage} % чтобы узнать номер последней страницы

\usepackage{enumitem} % дополнительные плюшки для списков
%  например \begin{enumerate}[resume] позволяет продолжить нумерацию в новом списке
\usepackage{caption}

\usepackage{url} % to use \url{link to web}



\newcommand{\smallduck}{\begin{tikzpicture}[scale=0.3]
    \duck[
        cape=black,
        hat=black,
        mask=black
    ]
    \end{tikzpicture}}

\usepackage{fancyhdr} % весёлые колонтитулы
\pagestyle{fancy}
\lhead{}
\chead{Локальный экстремум без производных}
\rhead{}
\lfoot{}
\cfoot{}
\rfoot{}

\renewcommand{\headrulewidth}{0.4pt}
\renewcommand{\footrulewidth}{0.4pt}

\usepackage{tcolorbox} % рамочки!

\usepackage{todonotes} % для вставки в документ заметок о том, что осталось сделать
% \todo{Здесь надо коэффициенты исправить}
% \missingfigure{Здесь будет Последний день Помпеи}
% \listoftodos - печатает все поставленные \todo'шки


% более красивые таблицы
\usepackage{booktabs}
% заповеди из докупентации:
% 1. Не используйте вертикальные линни
% 2. Не используйте двойные линии
% 3. Единицы измерения - в шапку таблицы
% 4. Не сокращайте .1 вместо 0.1
% 5. Повторяющееся значение повторяйте, а не говорите "то же"


\setcounter{MaxMatrixCols}{20}
% by crazy default pmatrix supports only 10 cols :)


\usepackage{fontspec}
\usepackage{libertine}
\usepackage{polyglossia}


\usepackage[bibencoding=auto, backend=biber, sorting=none, style=alphabetic]{biblatex}
\addbibresource{poly_opt_tricks.bib}


\setmainlanguage{russian}
\setotherlanguages{english}

% download "Linux Libertine" fonts:
% http://www.linuxlibertine.org/index.php?id=91&L=1
% \setmainfont{Linux Libertine O} % or Helvetica, Arial, Cambria
% why do we need \newfontfamily:
% http://tex.stackexchange.com/questions/91507/
% \newfontfamily{\cyrillicfonttt}{Linux Libertine O}

\AddEnumerateCounter{\asbuk}{\russian@alph}{щ} % для списков с русскими буквами
% \setlist[enumerate, 2]{label=\asbuk*),ref=\asbuk*}

%% эконометрические сокращения
\DeclareMathOperator{\Cov}{\mathbb{C}ov}
\DeclareMathOperator{\Corr}{\mathbb{C}orr}
\DeclareMathOperator{\Var}{\mathbb{V}ar}
\DeclareMathOperator{\col}{col}
\DeclareMathOperator{\row}{row}

\let\P\relax
\DeclareMathOperator{\P}{\mathbb{P}}

\DeclareMathOperator{\E}{\mathbb{E}}
% \DeclareMathOperator{\tr}{trace}
\DeclareMathOperator{\card}{card}
\DeclareMathOperator{\mgf}{mgf}

\DeclareMathOperator{\Convex}{Convex}
\DeclareMathOperator{\plim}{plim}

\DeclareMathOperator{\yan}{yan}

\usepackage{mathtools}
\DeclarePairedDelimiter{\norm}{\lVert}{\rVert}
\DeclarePairedDelimiter{\abs}{\lvert}{\rvert}
\DeclarePairedDelimiter{\scalp}{\langle}{\rangle}
\DeclarePairedDelimiter{\ceil}{\lceil}{\rceil}

\newcommand{\cN}{\mathcal{N}}
\newcommand{\cF}{\mathcal{F}}

\newcommand{\RR}{\mathbb{R}}
\newcommand{\NN}{\mathbb{N}}
\newcommand{\hb}{\hat{\beta}}
\newcommand{\dPois}{\mathrm{Pois}}





\begin{document}


\begin{verse}
    \begin{flushright}
        — Что мы знаем о лисе? \\
        — Ничего. И то не все. \\

        Борис Заходер
    \end{flushright}
\end{verse}

Цель этой заметки — рассказать, как искать локальный экстремум многочленов без производной.
В первой части изложен метод Глеба Весельского, придуманный им в клш в августе 2024.
Метод Глеба основан на подборе дополнительных сомножителей к неравенству среднего арифметического и среднего геометрического.
Метод годится для полинома любой степени, разложенного на линейные сомножители.
Во второй части изложен метод Иоганна Худде, мэра Амстердама в 17-м веке. 
Метод Худде основан на выявлении кратного корня с помощью  домножения коэффициентов многочлена на арифметическую прогрессию.


\section{Метод средних для нахождения экстремума}

\begin{tcolorbox}[colback=yellow!50!red!25!white]
Среднее арифметическое положительных чисел больше либо равно среднему геометрическому,
\[
\frac{x_1 + x_2 + \dots + x_n}{n} \geq (x_1 x_2 \dots x_n)^{1/n}.
\]
Точное равенство достигается, если
\[
x_1 = x_2 = \dots = x_n.
\]
\end{tcolorbox}

\subsection*{Пример попроще}

Найдите локальный максимум функции $f(x) = x^2 \cdot (6 - 2x)$.


Решение. Заметим, что $x + x + (6 - 2x) = 6$.
Значит, мы знаем среднее арифметическое,
\[
\frac{x + x + (6- 2x)}{3} = 2.
\]
В силу неравенства средних,
\[
\frac{x + x + (6- 2x)}{3} \geq (x\cdot x \cdot (6 - 2x))^{1/3}.
\]
Отсюда, если $x > 0$ и $6 - 2x>0$, 
\[
2^3 \geq x^2 \cdot (6 - 2x).
\]
При этом наибольшее значение достигается при $x = 6 - 2x$, то есть при $x=2$.


TODO: добавить картинку


\subsection*{Пример посложнее}


Найдите локальный максимум функции $f(x) = x (x + 1) (2 - x)$.


Хьюстон, у нас проблемы со старым решением!
В прошлом решении нам дважды повезло.
Во-первых, в простой задаче сумма сомножителей сама собой оказалась константой.
Во-вторых, в простой задаче было возможно одновременное равенство всех сомножителей между собой. 
В новой задаче ни сумма не равна константе, $x + (x + 1) + (2 - x) = 3 +x$, ни одновременное равенство $x = x+1 = 2-x$ не возможно.

Впрочем, точка оптимума не изменится, если мы домножим один сомножитель на $a > 0$, а второй — на $b > 0$.
С помощью подбора $a$ и $b$ мы вернём двойную удачу!

Найдём максиму функции $h(x) = ax \cdot b(x + 1) (2 - x)$.

Потребуем постоянную сумму и равенство сомножителей!
\[
\begin{cases}
    a + b - 1 = 0 \\
    ax = 2 - x \\
    b(x + 1) = 2 - x.
\end{cases}
\]

Величины $a$ и $b$ играют вспомогательную роль, избавимся от них, $a = (2 - x)/x$, $b = (2 - x)/(x + 1)$!
Получаем одно уравнение 
\[
\frac{2 - x}{x} + \frac{2 - x}{x + 1} - 1 = 0.
\]

Домножаем на знаменатели,
\[
(2 - x)(x + 1) + (2 - x) x - x (x + 1) = 0.
\]

Читатель, знакомый с производной, может заметить, что производную-то мы и получили!
\[
f'(x) = (2 - x)(x + 1) + (2 - x) x - x (x + 1).
\]

Остаётся лишь решить квадратное уравнение,
\[
(2 - x)(x + 1) + (2 - x) x - x (x + 1) = 0.
\]

Получаем $x_1 = (1 - \sqrt{7}) / 3$, $x_2 = (1 + \sqrt{7}) / 3$ — точки подозрительные на экстремум.


Аналогично можно найти и локальные экстремумы любого многочлена разложенного на сомножители. 
Для нахождения экстремума произвольного многочлена степени $n$ потребуется $n - 1$ дополнительный сомножитель. 
А частные случаи при везении могут решаться быстрее. 

TODO: добавить картинку


\subsection*{Примерчик с везением}

Найдите локальный максимум функции $f(x) = x^{10} (5 - x)$.

Домножим функцию на $10$ чтобы сделать сумму сомножителей постоянной.
\[
h(x) = x^{10}(50 - 10x).
\]

Воспользуемся неравенством средних, верным при $x > 0$ и $50 - 10x > 0$,
\[
\frac{x + x + \dots + x + (50 - 10x)}{11} \geq (x \cdot x \cdot \dots \cdot x \cdot (50 - 10x) )^{1/11}.
\]
Отсюда, при $x > 0$ и $50 - 10x > 0$,
\[
h(x) \leq (50/11)^{11}.
\]
Максимум достигается при $x = x = \dots = x = 50 - 10x$, то есть $x = 50/11$.

TODO: добавить картинку

\subsection*{Ещё примерчик}

Найдём экстремумы функции 
\[
f(x) = 4x + 3 + \frac{1}{x - 1}.
\]
Для удобства заменим $x - 1$ на $t$,
\[
g(t) = 4(t + 1) + 3 + \frac{1}{t} = 7 + 4t + \frac{1}{t}.
\]
По неравенству средних, находим локальный минимум,
\[
\frac{4t + 1/t}{2} \geq \sqrt{4t \cdot 1/t} = 2.
\]
Достигается равенство при $4t = 1/t$, то есть при $t = 1/2$ и $t = -1/2$.
Точка $t=1/2$ или $x = 1.5$ — это локальный минимум, а $t = -1/2$ или $x = 0.5$ — локальный максимум.


TODO: добавить картинку


\section{Метод Иоганна Худде с арифметическими прогрессиями}

Иоганн Худде жил в 17 веке, был математиком, мэром Амстердама и придумал метод нахождения локальных экстремумов без производных. 

Cнова начнём с простой задачки!

Найдите хотя бы один локальный экстремум функции
\[
f(x)  = (x - 2)^2 (x - 5) (x - 7)^4.
\]

В точке $x=5$ экстремума нет, так как при переходе переменной $x$ через $5$ функция меняет знак. 
А вот в точках $x = 2$ и $x = 7$ есть локальные минимумы. 
В этих точках функция равна нулю, а справа и слева от этих точек функция положительна. 
Функция достигает своего локального минимума $f_{\text{min}} = 0$ в точках кратных сомножителей!


TODO: добавить картинку


Впрочем, экстремум не обязательно равен нулю. 
Он может быть равен числу $m$.
Поэтому Иоганн Худде объявил охоту за кратными корнями уравнения $f(x) = m$.


\subsection*{Охота за кратными корнями}

Придумаем действие, которое бы упрощало многочлен $p(x) = f(x) - m$, 
но при этом любой кратный корень многочлена $p(x)$ оставался бы корнем нашего нового изобретаемого многочлена.

Начнём с параболы $p(x) = (x - 1)^2 = x^2 - 2x + 1$. 
Конечно, кратный корень у этой параболы равен $x = 1$.


Мне не известно, как Иоганн Худде пришёл к своей идее. 
Наверное, это были долгие часы безрезультатных поисков, а потом внезапное открытие. 
Далее лишь моя попытка реконструкции его рассуждений. 

Нам нужно строить новый многочлен $q(x)$ на базе старого $p(x) = x^2 - 2x + 1$.
Поэтому давайте домножим коэффициенты старого на произвольные константы, $q(x) = ax^2 - 2bx + c$. 
Хм, но при этом действии мы, конечно, можем потерять кратный корень! 
Поэтому давайте вернём кратный корень обратно! 
Для этого число $c$ не может быть произвольным, оно обязано подбираться так, чтобы кратный корень не погиб. 
В нашем игрушечном примере мы знаем, что кратный корень равен $x = 1$.
Поэтому новый многочлен должен иметь вид $q(x) = a(x - 1)^2 - 2b (x - 1)$.

Раскрываем скобки!
\[
    q(x) = a(x - 1)^2 - 2b (x - 1) = ax^2 - 2(a + b) x + (a + 2b).
\]
Внимательно смотрим на коэффициенты: $a$, $a + b$, $a + 2b$!
Они образуют арифметическую прогрессию!

Возможно, именно так и была открыта теорема.

Теорема (Иоганн Худде):
Если коэффициенты исходного многочлена домножить на любую арифметическую прогрессию,
то корнями нового многочлена обязательно будут все кратные корни исходного. 

Мы, конечно, её доказали только для квадратного многочлена с корнем $x=1$, но давайте поверим, что она верна для любого многочлена, и применим её в деле!

Найдите локальный максимум функции $f(x) = x (x + 1) (2 - x)$.

Раскрываем скобки, $f(x) = -x^3 + x^2 + 2x$.

Мы не знаем, чему равен экстремум, поэтому будем искать кратные корни многочлена 
\[
p(x) = -x^3 + x^2 + 2x - m.
\]
Домножим коэффициенты этого многочлена на прогрессию $(3, 2, 1, 0)$, чтобы избавиться от неизвестного $m$,
\[
\yan(p(x)) = -3x^3 + 2x^2 + 2x + 0.
\]
Решаем уравнение, $-3x^3 + 2x^2 + 2x + 0 = 0$, получаем корни $x_1 = 0$,  $x_2 = (1 - \sqrt{7}) / 3$, $x_3 = (1 + \sqrt{7}) / 3$.

Корень $x = 0$ возможен только при $m = 0$, при этом  $p(x) = x\cdot (-x^2 + x + 2)$ и корень $x = 0$ не кратный. 

Читатель, знакомый с производной, может заметить, что умножение на арифметическую прогрессию $\dots, 3, 2, 1, 0$ и сокращение сомножителя $x$ полностью эквивалентно взятию производной. 
\[
\yan(p(x)) = -3x^3 + 2x^2 + 2x + 0 = x \cdot (-3x^2 + 2x + 2) = x \cdot p'(x).
\]

Мощь метода Иоганна Худде состоит в том, что мы сами можем выбирать арифметическую прогрессию. 
И можем специально поставить в прогрессии ноль там, где в исходном многочлене стоит неприятный коэффициент.
Подробнее про метод можно прочитать в замечательной статье Джеффа Удзуки \cite{uzuki2005lost}.

\subsection*{Нахождение касательной}

Найдите касательную к графику $x^3$ в точке $x = 2$.

Мы хотим, чтобы корень $x = 2$ был кратным корнем уравнения $x^3 = kx + b$.

Перенесём всё в левую часть 
\[
p(x)  = x^3 + 0x^2 - kx - b = 0.
\]
Домножим на две разные прогрессии, $(2, 1, 0, -1)$ и $(3, 2, 1, 0)$.
Первая прогрессия избавит нас от параметра $k$, а вторая — от $b$.
Кстати, умножение на прогрессию $(0, 1, 2, 3)$ избавило бы нас от высоких степеней.
\[
\begin{cases}
    2x^3 + b = 0 \\
    3x^3 - kx = 0 \\
\end{cases}
\]
Мы хотим, чтобы кратным корнем был $x = 2$.
Из системы находим $b = - 16$, $k = 12$.


\subsection*{Ещё примерчик}

Найдём экстремумы функции 
\[
f(x) = 4x + 3 + \frac{1}{x - 1}.
\]
Для удобства заменим $x - 1$ на $t$,
\[
g(t) = 4(t + 1) + 3 + \frac{1}{t} = 4t + 7 + \frac{1}{t}.
\]
Приравняем эту функцию к ещё пока неизвестному значению экстремума $m$,
\[
4t + 7 + \frac{1}{t} = m.
\]
Домножим на $t$, 
\[
4t^2 + (7 - m) t + 1 = 0.
\]
Это уравнение должно иметь кратный корень, домножаем коэффициенты на прогрессию $(1, 0, -1)$,
\[
4t^2 - 1 = 0.
\]
Отсюда, $t = 1/2$ и $t = -1/2$.
Следовательно, $x = 1.5$ и $x = 0.5$.


\subsection*{Доказательство теоремы Иоганна Худде}

Во-первых, посмотрим, как операция $\yan()$ домножения коэффициентов многочлена на арифметическую прогрессию 
действует на самые простые многочлены с кратным корнем $x_0$:
\[
\yan(x^n(x - x_0)^2) = \yan(x^{n+2} - 2x^{n+1} x_0 + x^n x_0^2) = ax^{n+2} - 2(a + b)x^{n+1} x_0 + (a+2b)x^n x_0^2
\]
Подставляем кратный корень $x = x_0$ исходного многочлена  в преобразованный многочлен и замечаем, что получится ноль. 

Во-вторых, домножение коэффициентов многочлена на арифметическую прогрессию, $\yan()$, — это линейная операция:
\[
\yan(a(x) + b(x)) = \yan(a(x)) + \yan(b(x)).
\]
Во-третьих, любой многочлен с кратным корнем $x_0$ можно записать в виде
\[
p(x) = a_n x^n (x - x_0)^2 + a_{n-1} x^{n-1}(x - x_0)^2 + \dots + a_0 (x - x_0)^2.
\]
Применим операцию $\yan()$ к левой и правой части. 
В силу линейности применяем $\yan()$ к каждому слагаемому.
Один сомножитель $(x - x_0)$ выживает в каждом слагаемом, а значит и во всей сумме выживает корень, бывший кратным в изначальном многочлене.


\section*{Источники мудрости}
\addcontentsline{toc}{section}{Источники мудрости}
\printbibliography[heading=none]





\end{document}

